%!TEX root = ../../report.tex

 \subsubsection{Voxels [NOT \dots]} % (fold)
 \label{ssub:voxels}

``Voxel is a combination of ``volume" and ``pixel" ". i. e. the equivalent to pixels in 3D. In other words, if a pixel is a tiny square that represents a part of an model in a plane (picture), a voxel is a cube that represents a part of a model in a space.
Because we usually what we want to model  have three dimensions, it's easier to visualize this models with voxels.
We can see the world as a gigantic 3D matrix, and the voxels represent what is inside each point in the matrix.

Given that this voxels are just a way to store information, we still need a way to visualize this information. And this is the \trikyd part with this approach. How to visualize surfaces out of this 3D boxes. Since most of the hardware has evolved and beed adapted to work with textured triangles, and also a big volume of the research is arround textured triangles problems.
(\dots?)
% subsubsection voxels (end) 