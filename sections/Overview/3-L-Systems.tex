%!TEX root = ../../report.tex

\subsubsection{L-Systems} % (fold)
\label{ssub:l_systems}

Lindenmayer Systems (L-Systems) are a class of string rewriting mechanisms, originally developed by Lindenmayer as a mathematical theory for plant development.

One L-System is a type of a formal grammar and a string rewriting system that is capable of describe the behaviour of plant cells and model the growth processes of plant development.

It consists of two different parts, one axiom and a set of production rules. The axiom is the starting point of the system, acting as a seed. Then is't applied in this seed the set of production rules, that change the initial string and producing other strings.
This is an iterative process, so after the production of a larger set of strings, the rules can be applied to each one of them wish grows the size of the set even larger.

This L-Systems are used to produce natural growth of vegetation (Figure~\ref{fig:trees}), and the generation of Fractals. 


\begin{figure}[htbp]
    \centering
    \includegraphics[width=0.95\textwidth]{img/Theory/L_Systems/Dragon_trees.jpg}
    \caption{Trees with L-Systems}
    \label{fig:trees}
\end{figure}


In this process, each symbol is associated with a production rule. For instance having $\{F, +, -\}$ for the alphabet and \emph{production} $\{F \rightarrow
 F+F--F+F\}$. From a starting axiom \emph{aba}, and the application of the rules we have:\\
\begin{equation} \label{eq:seed}
F\\
\end{equation}
\begin{equation} \label{eq:step1}
F+F--F+F\\
\end{equation}
\begin{equation} \label{eq:step2}
F+F--F+F \; + \; F+F--F+F \;- \;- \;F+F--F+F \;+ \;F+F--F+F\\
\end{equation}

%\begin{align}
%\begin{split}
%F\\
%F+F--F+F\\
%F+F--F+F \; + \; F+F--F+F \;- \;- \;F+F--F+F \;+ \;F+F--F+F\\
%\end{split}
%\end{align}
%\\

This is an example of the evolution of one system where the production is applied  in (\ref{eq:seed}) that turns into $F+F--F+F$. In Note that the space between the symbols are just for readability.

All the symbols are assigned with a geometric meaning. The notion of a turtle with a pen, as proposed in \cite{abelson1982aa}, with the symbols being interpreted as moving instructions to the turtle, is a simple way to understand. If ``F'' means forward and the symbols ``+'' and ``-'' are interpreted as rotations counter-clockwise and clockwise respectively by a predefined angle. 

Using the given example, and setting the angle for the rotation to $60^{\circ}$ the result is Figure~\ref{fig:kockLS}.
%$\bigodot \; \bigodot$

\begin{figure}[htbp]
   \centering
   \includegraphics[width=0.75\textwidth]{img/Theory/L_Systems/koch.png}
   \caption{}
   \label{fig:kockLS}
\end{figure}


% This concept of the turtle can be considered also in 3D.



% subsubsection l_systems (end)