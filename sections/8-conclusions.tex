%!TEX root = ../report.tex

% 
% Conclusions
% 

\section{Conclusions}
\label{sec:conclusions}

Architects and designers increasingly use programming as a tool which helps them to develop their work. This powerful tool enables them to create faster and with greater creative freedom. With the development of their programming capabilities they begin to create larger, more complex models and to test the technology at maximum and thus realize their limitations.

The problem of performance arises because the systems being used were not built with this problem in mind. These were developed for a manual, slow usage. It was not thought at the time that a user could generate massive amounts of geometry in seconds. As a result users have to wait for large periods of time since they run their programs until they can see a result.

Therefore there is a need to solve this problem, therefore this work is also needed given that proposes a solution that puts limits on another level allowing users to create.

This solution must have a good performance and support the most used functions that the users need to become a valid alternative to the currently used tools.

My solution shortcuts the Rosetta traditional pipeline to improve performance, doing that by using Rosetta just as an interface and transferring as most of the processing as possible to the GPU.

One prototype is implemented that already creates boxes and cylinders without transformations that allows the creation of the city example in Figure~\ref{fig:pic1}.

In the future, I will explore how to implement the rest of the primitives without losses in performance and how to introduce rotation to the objects without adding much data to the current primitive description set.