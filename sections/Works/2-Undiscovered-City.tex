%!TEX root = ../../report.tex

\subsection{Undiscovered City} % (fold)
\label{sub:undiscovered_city}

In \cite{Greuter2003} Stefan Greuter et al. presented a system that generates in real-time pseudo infinite virtual cities which can be interactively explored from a first person perspective. In their approach ``all geometrical components of the city are generated as they are encountered by the user." As shown in the Figure~\ref{fig:viewingRange} only the part of city that is inside the viewing range is generate. This method allows the visualization of massive amounts of geometry, buildings in this case, by generating in real time only the geometry that on sight, and since this subset is usually much smaller than all the geometry this results in huge benefits in performance.

\begin{figure}[htbp]
	\centering
	\includegraphics[width=0.85\textwidth]{img/Real-Time-procedural-generation/viewing-range.png}
	\caption{Viewing Range}
	\label{fig:viewingRange}
\end{figure}

\subsubsection{Road Network} % (fold)
\label{ssub:road_network}

The system uses a 2D grid that divide the terrain into square cells. The cells represent proxies for the content that will be procedurally generated. Before the content of each cell is generated, the potential visibility of it is tested, and after that, only the visible cells are filled with content.

Then the roads are created in a uniform grid pattern. This grid does not feel very natural, and in the continuation of the work, this system evolved into a more realistic one with the join of some of the grids to create a less uniform distribution of the buildings.

% subsubsection road_network (end)

\subsubsection{Buildings} % (fold)
\label{ssub:buildings}


To compute the form and appearance of each building, it is used a ``single 32 bit pseudo random number generator seed. The random sequence determines building properties such as width, height and number of floors."
Similar sequences of number result in similar buildings. To avoid that, it is used a a hash function to convert each cell position into a seed.

To generate a building the first is to generate a floor plan. To do so, it is randomly selected and merged a set of regular polygons and rectangles, then this is extruded. This is an iterative process, that creates sections from the top to the bottom, by adding more shapes to the the initial shape and extruding as shown in the Figure~\ref{fig:UC_buildings}. Starting from the left, first there is a simple polygon, that is merged with a rectangle and after extrusion, forms the first block that will be the top of the building. After that, another extrusion is made to generate the next block followed by the merge of a rectangle to the floor shape and the generation of a new block and so on.

\begin{figure}[htbp]
	\centering
	\includegraphics[width=0.85\textwidth]{img/Real-Time-procedural-generation/Building-Generation.png}
	\caption{buildings}
	\label{fig:UC_buildings}
\end{figure}

With the application of this method very complex architectural forms can be generated, depending only on which forms are selected and the order that is used to merge them.

% subsubsection buildings (end)

% subsection undiscovered_city (end)