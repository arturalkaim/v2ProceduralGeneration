%!TEX root = ../report.tex

% 
% Abstract 
% 

\begin{abstract}

Existing tools used for 3D modeling creation are mostly geared for manual use. Unfortunately, the manual production of large amounts geometry is very time consuming. Procedural generation of these forms is one of the approaches which considerably speeds up this process. This approach consists in the algorithmic construction of these forms and allows the quick creation of massive amounts of geometry. As most 3D modeling tools were not made specifically for this type of use, favoring instead manual use, they do not have the performance necessary for a smooth use. This work proposes solutions to this performance problem, through the use of different techniques that accelerate the production and visualization of large volumes of geometry.


%The existing graphic creation tools are geared for manual use. Unfortunately, the manual production of large amounts of complex architectural forms is very time consuming. Procedural generation of these forms is one of the approaches which considerably speed up this process. This approach consists in the algorithmic construction of these forms through different techniques like shape grammars, L-Systems, etc. This project aims to explore this approach and apply it to the procedural modeling of cities by the development of a tool that, in connection with the Rosetta tool, will provide a new mean for the users to explore this approach. (\dots)


\end{abstract}

COMENT\'ARIOS DA DISCUSSAO

\begin{itemize}
	\item O número de polígonos é muito baixo. Temos de ver o que é preciso para aumentar substancialmente o número de polígonos. \textbf{??}

	\item Tornar mais claro como é que se passa da geração procedimental para a produção de geometria no GPU. \textbf{??}
	
	\item A ordem das secções da tese devia ser a mesma da apresentação. \Checkmark

	\item Colocar os objectivos perto do início. \Checkmark

	\item Background do OpenGL devia estar no início. \Checkmark

	\item Clarificar os vários níveis do occlusion culling. \textbf{??}

	\item Falta tabela que mostre quais as técnicas usadas pelas várias ferramentas apresentadas no related work. \textbf{??}

	\item Quem vai usar as técnicas procedimentais são os arquitectos (l-systems, autómatos celulares, etc) que, por isso, vão gerar muita geometria. Essa geometria vai ser reduzida a dados elementares que depois alimentam o GPU. \Checkmark \textbf{??}

	\item Pensar melhor na avaliação para não parecer injusto. Eventualmente comparar com o City Engine. \textbf{??}
\end{itemize}
