%!TEX root = ../report.tex

% 
% Abstract 
% 

\begin{abstract}

The existing graphic creation tools are geared for manual use. Unfortunately, the manual production of large amounts geometry is very time consuming. Procedural generation of these forms is one of the approaches which considerably speed up this process. This approach consists in the algorithmic construction of these forms and allows the creation of massive amounts of geometry very fast. As these tools were not made specifically for this type of use, favoring manual use, they do not have the high performance necessary for smooth use as it takes a long time since we run the program until you can see the result. This work proposes a solution for this performance problem, through the use of different techniques and technologies to accelerate the visualization of large amounts of geometry.


%The existing graphic creation tools are geared for manual use. Unfortunately, the manual production of large amounts of complex architectural forms is very time consuming. Procedural generation of these forms is one of the approaches which considerably speed up this process. This approach consists in the algorithmic construction of these forms through different techniques like shape grammars, L-Systems, etc. This project aims to explore this approach and apply it to the procedural modeling of cities by the development of a tool that, in connection with the Rosetta tool, will provide a new mean for the users to explore this approach. (\dots)

\end{abstract}