%!TEX root = ../report.tex

% 
% Related work
% 

\section{Related Work}
\label{sec:related_work}

In this section I give an overview of the related work that has been carried out on procedural generation of large amounts of geometry followed by a conclusion. After that, in the following sections, are explained some techniques of procedural modeling.
Since the target audience of this work are architects and designers, most of the work that I present have the same target.

The main goal of this systems is to generate large models. Consequently, this works face some of the same problems I face so it is important to learn about their solutions. 


%!TEX root = ../../report.tex

\subsection{CityEngine \cite{Parish2001} \cite{Muller2006}}
\label{sub:cityengine}

CityEngine is a three-dimensional (3D) modeling software developed by Procedural Inc. (now part of the Esri R\&D Center). it is specialized in the generation of 3D urban environments. With the procedural modeling approach, CityEngine enables an efficient creation of detailed and large-scale 3D city models with a lot of control from the user. This system applies the concept of Immediate Feedback by allowing the user to immediately see the results of each change. This is implemented through a set of sliders that are assigned to various indicators in the model that can be as high level as the \emph{size of the city} or as specific as the width of the windows or the number of floors in a building.

The following sections explain how CityEngine faces each of the steps in the generation of a city, such as road network generation or building modeling.

\subsubsection{RoadNetwork} % (fold)
\label{ssub:roadnetwork1}


The first part to procedurally generate a city is to create a road network to become a backbone of the city and provide an overall structure. For that, CityEngine receives as input maps such as land-water boundaries and population
density. From that input a network of highways is created to connect the areas off high density population, and small roads connect to the highways.
This growth process continues until the average area of each lot is the desired one. The system have a default value, but it can be set by the user to a different one.

To implement this growth process, it uses an L-System (Section~\ref{ssub:l_systems}) that computes the road network.


\begin{figure}[htbp]
  \centering
  \includegraphics[width=0.5\textwidth]{img/Procedural-Modeling-of-Cities/Capturar.png}
  \caption{Road Map growth}
  \label{fig:city}
\end{figure}

The Figure~\ref{fig:city} shows the evolution of this process in a map of Manhattan. The first two pictures on the top shows the process in different phases during the process, the picture in the middle is the result of the process and the bottom line is the real map of Manhattan for comparison.

% subsubsection roadnetwork (end)Road Network

\subsubsection{Buildings} % (fold)
\label{ssub:buildings1}

To implement the generation of buildings, has been created CGA, which is a shape grammar(Section~\ref{ssub:shape_grammars}) that was introduced by Pascal Muller, Peter Wonka and others, in \cite{Parish2001}. It is defined as ``a novel shape grammar for the procedural modelling of CG architecture, produces building shells with high visual quality and geometric detail." To do so, this grammar uses a group of well defined production rules.

This tool allows the user to model buildings with an high control and in different ways. It can be done by text, writing production rules from a shape grammar or with a visual language like Grasshopper 3D, that is nice for simple works but it is impossible to work with a slightly more complex work. 

\paragraph{Mass Modeling} % (fold)
\label{par:mass_modeling}
To model a building the first step is to create a mass model of the entire building by assembling basic shapes. With scaling, translation rotation and split applied to basic shapes namely I, L, H, U and T as shown in the Figure~\ref{fig:basic_shapes}.

\begin{figure}[htbp]
  \centering
  \includegraphics[width=0.95\textwidth]{img/Procedural-Modeling-of-Cities/MassModeling2.png}
  \caption{Basic shapes}
  \label{fig:basic_shapes}
\end{figure}

% paragraph mass_modeling (end)

The next step is to add the roof, from a set of basic roof shapes or general L-Systems.

After that, with the application of the grammar rules in the created mass, it is possible to create complexity to the level that is desired, being able to produce highly complex buildings like the one in Figure~\ref{fig:CEnewbuilding}.


\begin{figure}[htbp]
  \centering
  \includegraphics[width=0.8\textwidth]{img/Procedural-Modeling-of-Cities/building2.png}
  \caption{Complex building modeled with CGA}
  \label{fig:CEnewbuilding}
\end{figure}

% subsubsection buildings (end)

\subsubsection{Cities} % (fold)
\label{ssub:Cities1}

The result can be a city like Figure~\ref{fig:bigCity}, with approximately 26000 buildings.

\begin{figure}[htbp]
  \centering
  \includegraphics[width=0.95\textwidth]{img/Procedural-Modeling-of-Cities/City.png}
  \caption{City with approximately 26000 buildings.}
  \label{fig:bigCity}
\end{figure}

City Engine outputs can be imported by Maya\footnote{\url{http://www.autodesk.com/products/maya/overview}}, to achieve better results. Like the Figure~\ref{fig:cityMaya}, that represents a ‘virtual’ Manhattan.

\begin{figure}[htbp]
  \centering
  \includegraphics[width=0.95\textwidth]{img/Procedural-Modeling-of-Cities/City_Maya.png}
  \caption{City rendered with Maya.}
  \label{fig:cityMaya}
\end{figure}

% subsubsection subsubsection_name (end)

%!TEX root = ../../report.tex

\subsection{Undiscovered City} % (fold)
\label{sub:undiscovered_city}

In \cite{Greuter2003} Stefan Greuter et al. presented a system that generates in real-time pseudo infinite virtual cities which can be interactively explored from a first person perspective. In their approach ``all geometrical components of the city are generated as they are encountered by the user." As shown in the Figure~\ref{fig:viewingRange} only the part of city that is inside the viewing range is generate. This method allows the visualization of massive amounts of geometry, buildings in this case, by generating in real time only the geometry that on sight, and since this subset is usually much smaller than all the geometry this results in huge benefits in performance.

\begin{figure}[htbp]
	\centering
	\includegraphics[width=0.85\textwidth]{img/Real-Time-procedural-generation/viewing-range.png}
	\caption{Viewing Range}
	\label{fig:viewingRange}
\end{figure}

\subsubsection{Road Network} % (fold)
\label{ssub:road_network}

The system uses a 2D grid that divide the terrain into square cells. The cells represent proxies for the content that will be procedurally generated. Before the content of each cell is generated, the potential visibility of it is tested, and after that, only the visible cells are filled with content.

Then the roads are created in a uniform grid pattern. This grid does not feel very natural, and in the continuation of the work, this system evolved into a more realistic one with the join of some of the grids to create a less uniform distribution of the buildings.

% subsubsection road_network (end)

\subsubsection{Buildings} % (fold)
\label{ssub:buildings}


To compute the form and appearance of each building, it is used a ``single 32 bit pseudo random number generator seed. The random sequence determines building properties such as width, height and number of floors."
Similar sequences of number result in similar buildings. To avoid that, it is used a a hash function to convert each cell position into a seed.

To generate a building the first is to generate a floor plan. To do so, it is randomly selected and merged a set of regular polygons and rectangles, then this is extruded. This is an iterative process, that creates sections from the top to the bottom, by adding more shapes to the the initial shape and extruding as shown in the Figure~\ref{fig:UC_buildings}. Starting from the left, first there is a simple polygon, that is merged with a rectangle and after extrusion, forms the first block that will be the top of the building. After that, another extrusion is made to generate the next block followed by the merge of a rectangle to the floor shape and the generation of a new block and so on.

\begin{figure}[htbp]
	\centering
	\includegraphics[width=0.85\textwidth]{img/Real-Time-procedural-generation/Building-Generation.png}
	\caption{buildings}
	\label{fig:UC_buildings}
\end{figure}

With the application of this method very complex architectural forms can be generated, depending only on which forms are selected and the order that is used to merge them.

% subsubsection buildings (end)

% subsection undiscovered_city (end)


\subsection{Conclusion} % (fold)
\label{sub:conclusion}

This works present ways to generate and visualize large amounts of geometry, in this cases applied to urban models. While CityEngine \cite{Parish2001} aims to allow the users to create large and realistic urban models, where they give, in the limit, total control to the users. Undiscovered City\cite{Greuter2003} is much more a visualization tool, it generates the model automatically for the user to explore.

From these works there are some ideas to explore. The idea of immediate feedback that is implemented in CityEngine, with sliders, is a good input to my work. This helps the unexperienced users to easily see the how their code impact the results. Also the commands they have on the grammar could be an helpful inspiration for the design of my API.

From the Undiscovered City system, since they also have massive amount of geometry, how they tackle this problem is very important source of inspiration as well.

\subsection{Procedural Modeling Techniques} % (fold)
\label{sub:procedural_modeling_techniques}

In the following sections are explained some procedural modeling techniques. These techniques are applied to the generation of various types of forms procedurally and will be explored for application within this work.

% subsection procedural_modeling_techniques (end)

%!TEX root = ../../report.tex

\subsubsection{Fractals} % (fold)
\label{ssub:fractals}


A fractal is defined in \cite{Ebert2002} as ``a geometrically complex object, the complexity of which arises through the repetition of a given form over a range of scales''.
This concept is observed in some forms that exist in nature. Trees, mountains, coastlines and the network of neurons on a human cortex can be seen as examples of fractals. Natural shapes tend to be irregular and fragmented and exhibit a complexity incomparable to regular geometry \cite{mandelbrot1984fractal}.
Fractals were proposed to be seen as a new form of symmetry \cite{Ebert2002}, \emph{Dilation Symmetry}, which is when an object is invariant over a change of scale. This invariance might be only qualitatively and not exact. For instance, a river network exhibit dilation symmetry if \textit{zooming in} in some part looks the same as the whole image. As this example, many others show dilated symmetry such as clouds, tree branches and some vegetables as shown in Figure~\ref{fig:NFractals}. 

\begin{figure}
        \centering
        \begin{subfigure}[b]{0.4\textwidth}
                \includegraphics[width=\textwidth]{img/Theory/Fractals/Leaf.png}
                \label{fig:Fleaf}
        \end{subfigure}%
        ~ %add desired spacing between images, e. g. ~, \quad, \qquad, \hfill etc.
          %(or a blank line to force the subfigure onto a new line)
        \begin{subfigure}[b]{0.4\textwidth}
                \includegraphics[width=\textwidth]{img/Theory/Fractals/Fractal_Broccoli.jpg}
                \label{fig:Fbrocoli}
        \end{subfigure}
        \caption{Fractals in Nature}\label{fig:NFractals}
\end{figure}


This idea was applied in maths and resulted in a new area in this science called fractal mathematics. The objective of this field is to describe very complex shapes with simple rules such as repeating a substitution pattern. 

\begin{figure}[htbp]
	\centering
	\includegraphics[width=0.7\textwidth]{img/Theory/Fractals/Fractal1_1000.png}
	\caption{Geometric Fractals}
	\label{fig:GFractals}
\end{figure}

In Figure~\ref{fig:GFractals} there are four examples of Geometric Fractals, with the first five iterations of each one. All of them are built by the substitution of a part of the image by another one. 

The example of the second row is known as the Koch snowflake. In this example, at each iteration, all the line segments are replaced by four segments with 1/3 of the size of the original one with the two in the middle being placed in a angle forming a equilateral triangle with the original line that is removed.

It is clear that the detail that is presented in each iteration increases as the scale changes. There is the concept of \emph{Fractal Dimension} that tries to measure this evolution, in which the detail in a pattern changes in comparison with the scale in which it is measured.

As stated before, the world is visually very complex, so when synthesizing worlds, ``\emph{complexity} equals \emph{work}''\cite{Ebert2002}. This work can be done by the programmer/artist or by a computer. Fractals as being defined as a simple mathematical function, it is relatively easy to implement a procedure that model one fractal. 

This technique is used to model many natural forms that present fractal properties. Mountains, for instance, are usually modeled using of fractals. Other natural forms that present fractal properties are trees, river systems, lightning or vascular systems in living beings.




% subsection fractals (end)


%!TEX root = ../../report.tex

\subsubsection{Cellular Automaton} % (fold)
\label{sub:cellular_automaton}

A cellular Automaton is a model of a system of cells within a grid with a given shape, each of this cells can be on one of a finite set of states. It evolves during a finite amount of time steps with a set of simple rules according to each state of the neighboring cells.
The neighborhood of the cell can be defined in many different ways, the most common is the use of the adjacent cells. 

This models have various applications, such as modeling of nature aspects (Figure~\ref{fig:CArule30shell}), textures, and as inspiration to architecture (Figure~\ref{fig:CAarchitecture}).

\begin{figure}
        \centering
        %\begin{subfigure}[b]{0.5\textwidth}
                \includegraphics[width=0.45\textwidth]{img/Theory/Cellular_A/dome1.jpg}
        %        \caption{}
  %              \label{fig:CAdome}
        %\end{subfigure}%
        %~ %add desired spacing between images, e. g. ~, \quad, \qquad, \hfill etc.
          %(or a blank line to force the subfigure onto a new line)
          ~~
        %\begin{subfigure}[b]{0.5\textwidth}
                \includegraphics[width=0.45\textwidth]{img/Theory/Cellular_A/main1.jpg}
        %        \caption{}
   %             \label{fig:CArule30}
        %\end{subfigure}
        \caption{Examples of cellular automata applied in architecture}
        \label{fig:CAarchitecture}
\end{figure}

The case where each cell have two possible states and the next generation state depends only on the previous state of the cell and the two immediate neighbors is called an \emph{elementary cellular automaton}. In this case we have $2^3 = 8$ possible patterns for a neighborhood and $2^8 = 256$ sets of possible different rules. This rules are referred by their \emph{Wolfram code} \cite{CellularAutWOLFRAM}. 

A common initial state for this elementary cellular automata is a random line. But to be able to compare the results between rules and get clean results another option is to start with a line with zeros except the middle cell that is initialized with the value one. Applying this second option and the set of rules in Figure~\ref{fig:CArule} (the rule 30), we get the pattern in the Figure~\ref{fig:resultCA} that represents the evolution of a cellular automaton over a few generations.

\begin{figure}[htbp]
	\centering
	\includegraphics[width=0.85\textwidth]{img/Theory/Cellular_A/Rules.png}
	\caption{Example Production Rules\cite{Shiffman2012}}
	\label{fig:CArule}
\end{figure}



\begin{figure}[H]
    \centering
    \includegraphics[width=0.75\textwidth]{img/Theory/Cellular_A/Result.png}
    \caption{Sierpiński Triangle, rule 90}
    \label{fig:resultCA}
\end{figure}


In Figure~\ref{fig:resultCA} each line represents an iteration of the system with the application of the rules. With this set of rules a Sierpiński triangle is reproduced.


Cellular automata are used mainly to model phenomena that occur in the physical world, most of them can only express the basic idea of a phenomenon, but some are accurate enough to be able to make predictions.

In this context, cellular automata are used to model natural shapes and textures, Figure~\ref{fig:CArule30shell} shows on the left, a natural texture on the shell of a \emph{Textile Cone Snail}, that looks like the patterns formed with the cellular automaton on the right.



\begin{figure}
        \centering
        %\begin{subfigure}[b]{0.3\textwidth}
                \includegraphics[width=0.45\textwidth]{img/Theory/Cellular_A/shell.jpeg}
        %        \caption{a)}
		%		\label{fig:CAshell}
        %\end{subfigure}%
        %~ %add desired spacing between images, e. g. ~, \quad, \qquad, \hfill etc.
          %(or a blank line to force the subfigure onto a new line)
          ~~
        %\begin{subfigure}[b]{0.5\textwidth}
                \includegraphics[width=0.45\textwidth]{img/Theory/Cellular_A/Rule30.png}
		%		\caption{b)}
		%		\label{fig:CArule30}
        %\end{subfigure}
        \caption{Example of the representation of natural patterns with cellular automata. On the left, a Natural Shell \cite{Shiffman2012} and on the right a Pattern formed with the rule 30}
		\label{fig:CArule30shell}
\end{figure}



% subsection cellular_automaton (end)

%!TEX root = ../../report.tex

\subsubsection{L-Systems} % (fold)
\label{ssub:l_systems}

Lindenmayer Systems (L-Systems) are a class of string rewriting mechanisms, originally developed by Lindenmayer as a mathematical theory for plant
development. It is capable of describe the behavior of plant cells and model the growth processes of plant development.

An L-Systems consists of two different parts, one axiom and a set of production rules. The axiom is the starting point of the system, acting as a seed. Then it is applied in this seed the set of production rules, that change the initial string, producing other strings.
This is an iterative process, so after the production of a larger set of strings, the rules can be applied to each one of them which grows the size of
the set even more.

L-Systems are used to model the natural growth of vegetation (Figure~\ref{fig:trees}), and the generation of Fractals. 

\begin{figure}[htbp]
    \centering
    \includegraphics[width=0.65\textwidth]{img/Theory/L_Systems/Dragon_trees.jpg}
    \caption{Trees with L-Systems}
    \label{fig:trees}
\end{figure}


In this process, each symbol is associated with a production rule. For instance having $\{F, +, -\}$ for the alphabet and \emph{production} $\{F \rightarrow
 F+F--F+F\}$. From a starting axiom \emph{aba}, and the application of the rules we have:\\
\begin{equation} \label{eq:seed}
F\\
\end{equation}
\begin{equation} \label{eq:step1}
F+F--F+F\\
\end{equation}
\begin{equation} \label{eq:step2}
F+F--F+F \; + \; F+F--F+F \;- \;- \;F+F--F+F \;+ \;F+F--F+F\\
\end{equation}

%\begin{align}
%\begin{split}
%F\\
%F+F--F+F\\
%F+F--F+F \; + \; F+F--F+F \;- \;- \;F+F--F+F \;+ \;F+F--F+F\\
%\end{split}
%\end{align}
%\\

This is an example of the evolution of one system where the production is applied  in (\ref{eq:seed}) that turns into $F+F--F+F$. Note that the space
between the symbols are just for readability.

All the symbols are assigned with a geometric meaning. The notion of a turtle with a pen, as proposed in \cite{abelson1982aa}, with the symbols being
interpreted as moving instructions to the turtle, is a simple way to understand, where ``F'' means forward and the symbols ``+'' and ``-'' are interpreted as rotations counter-clockwise and clockwise respectively by a predefined angle. By applying this method to the last example and setting the angle for the rotation to $60^{\circ}$ the result is Figure~\ref{fig:kockLS}.

\begin{figure}[htbp]
   \centering
   \includegraphics[width=0.55\textwidth]{img/Theory/L_Systems/koch.png}
   \caption{Result of the ``turtle walk'' with the given example}
   \label{fig:kockLS}
\end{figure}

%\begin{wrapfigure}{r}{0.5\textwidth}
%	\vspace{-15pt}
%    \centering
%    \includegraphics[width=0.55\textwidth]{img/Theory/L_Systems/koch.png}
%    \caption{}
%    \label{fig:kockLS}
%	\vspace{-25pt}
%\end{wrapfigure}

%$\bigodot \; \bigodot$




% This concept of the turtle can be considered also in 3D.



% subsection l_systems (end)

%!TEX root = ../../report.tex

\subsubsection{Shape Grammars} % (fold)
\label{ssub:shape_grammars}


Shape Grammars can be considered grammars for design. Instead of having symbols or letters as components of the alphabet, it has shapes that can be in 2D or 3D, and has production rules that are composed by these shapes, that specify the evolution of the system. With this process, similar to the L-Systems explained before, the shape starts from a seed, i.e. a usually simple shape and can evolve to one big and/or complex shape.

The process is performed in two steps, the recognition of a shape and the replacement according to the rules previously defined. 

Figure~\ref{fig:SGrammars} exemplifies one shape grammar with one rule, and the evolution of the application of this rule to the shapes iteratively. In this image, it is shown that from very simple initial shape, a complex from can be generated after a few iterations.

\begin{figure}
        \centering
		\begin{subfigure}[b]{0.55\textwidth}
			\includegraphics[width=\textwidth]{img/Theory/Shape_Grammars/Grammar.png}
			\caption{a)}
			\label{fig:SGGrammar}
		\end{subfigure}
        
         %add desired spacing between images, e. g. ~, \quad, \qquad, \hfill etc.
          %(or a blank line to force the subfigure onto a new line)
		\begin{subfigure}[b]{0.55\textwidth}
			\includegraphics[width=\textwidth]{img/Theory/Shape_Grammars/Recursion.png}
			\caption{b)}
			\label{fig:SGRecursion}
		\end{subfigure}
        \caption{a) Grammar Tiles b) Recursion steps}
        \label{fig:SGrammars}
\end{figure}

In the CityEngine \cite{Muller2006} system, this is applied to the generation of buildings using 3D blocks for the main form, and 2D shapes to design the facades.

% \begin{figure}[htbp]
% 	\centering
% 	\includegraphics[width=0.55\textwidth]{img/Theory/Shape_Grammars/Edificio.png}
% 	\caption{Simple Building}
% 	\label{fig:SGBuilding}
% \end{figure}

Figure~\ref{fig:SGBuilding} shows a simple building that I modelled using CityEngine and its CGA Shape Grammar (Section~\ref{sub:cityengine}). But CGA is powerful enough to model much more complex buildings like the one in Figure~\ref{fig:CEBuilding}.


% \begin{figure}[htbp]
% 	\centering
% 	\includegraphics[width=0.55\textwidth]{img/Theory/Shape_Grammars/Capturar.png}
% 	\caption{Complex Building \cite{Muller2006}}
% 	\label{fig:CEBuilding}
% \end{figure}

\begin{figure}
\centering
\begin{minipage}{.5\textwidth}
  \centering
  \includegraphics[width=.65\linewidth]{img/Theory/Shape_Grammars/Edificio.png}
  \captionof{figure}{Simple Building}
  \label{fig:SGBuilding}
\end{minipage}%
\begin{minipage}{.5\textwidth}
  \centering
  \includegraphics[width=1.1\linewidth]{img/Theory/Shape_Grammars/Capturar.png}
  \captionof{figure}{Complex Building \cite{Muller2006}}
  \label{fig:CEBuilding}
\end{minipage}
\end{figure}

% subsubsection shape_grammars (end)

%%!TEX root = ../../report.tex

\subsubsection{Tiling} % (fold)
\label{ssub:tiling}

A common solution to give realism to 3D objects is the application of textures. One problem is that this textures take a lot of memory space. To work around this problem an easy solution is to have a small texture piece, a tile, and repeat it throughout the objects, this is called tiling. Or as defined in \cite{TilingWOLFRAM}: ``A plane-filling arrangement of plane figures or its generalisation to higher dimensions.".  This means, the result of constructing a plane from a finite set of ``tiles". 

If this technique is used naively, commonly result in not very homogeneous textures. It depends much on the set of tiles that are used. If the borders of all the tiles are all the same the result is always homogeneous. Increase if the borders are very different the chances of getting a non-uniform texture rises. Figure~\ref{fig:TIrregulartexture}, from \cite{ProcWorld} shows how a bad structured tiling system produces a non-homogeneous texture.  


\begin{figure}[htbp]
	\centering
	\includegraphics[width=0.95\textwidth]{img/Theory/Tiling/iregular.png}
	\caption{One tile and an irregular pattern}
	\label{fig:TIrregulartexture}
\end{figure}


To make uniform planes, the boundary of each tile must be coherent, i. e., the borders of connecting tiles have to match. Given a single tile, the so-called first corona is the set of all tiles that have a common boundary point with the tile (including the original tile itself). From that, the simplest method to create a homogeneous texture is to connect each tile with one that belongs to its first corona.

The easy, most simple solution is to make sure that all the tiles have the same borders all around. With this property it is guaranteed that any created texture will be homogeneous. But this solution does not provide much irregularity and the results present repeating patterns. 



\emph{Wang Tiles \cite{Cohen2003}} is a solution named after Mr Hao Wang that predicted that tiling was not possible. It received his name, not only for him being wrong, but because the proof that this is possible uses much of the work he did trying to prove the impossibility. This process allows tiling with an arbitrary number of different vertical and horizontal borders and from that calculate the set of tiles that are needed to create a full texture without inconsistencies. 



\begin{figure}
        \centering
		%\begin{subfigure}[b]{0.4\textwidth}
			\includegraphics[width=0.45\textwidth]{img/Theory/Tiling/tiles.png}
		%	\caption{a)}
		%	\label{fig:TTileSet}
		%\end{subfigure}
        ~ ~%add desired spacing between images, e. g. ~, \quad, \qquad, \hfill etc.
          %(or a blank line to force the subfigure onto a new line)
		%\begin{subfigure}[b]{0.4\textwidth}
			\includegraphics[width=0.45\textwidth]{img/Theory/Tiling/plane.png}
		%	\caption{b)}
		%	\label{fig:TPlanePortion}
		%\end{subfigure}
        \caption{a) Eight Wang Tiles b) Portion of a plane}
        \label{fig:WangTiles}
\end{figure}

So the process is to assign colors to the tiles borders and then matching colored borders are aligned forming a plane.

% paragraph wang_tiles_ (end)

As you might have noticed, the inner content of the tiles are not a problem. As we are trying to create the uniform textures by arranging this smaller pieces only the borders matter, so we can create a set of tiles with the same borders and whatever inner content we want. With this technique the result can be much more irregular and more natural looking textures.


\emph{Corner Tiles \cite{LD06AWTCECC}}. The vanilla wang tiles have problems in the diagonals that are not taken into account. They are the confrontation between four tiles which leads to less homogeneous texture if we the borders do not match. By using the corners, the problem goes to the sides, that despite being larger, are only the confrontation between two tiles and therefore it leads to a more homogeneous texture.
 % paragraph corner_tiles (end) 


% \paragraph{Genetic tile generation} % (fold)
% \label{par:genetic_tile_generation}
% “The bottom line for me is, Wang tiles are amazing things until you try to use them seriously. They work great for stuff you can synthesise from the ground up. If you are trying to mix samples from real life, get ready for some trouble.” \cite{ProcWorld}
%\url{http://procworld.blogspot.pt/2013/01/tile-genetics.html}
% paragraph Genetic tile generation (end)




% subsubsection tiling (end)

%!TEX root = ../../report.tex

\subsection{Noise} % (fold)
\label{ssub:noise}


``To generate irregular procedural textures, we need an irregular primitive function, usually called noise" \cite{Ebert2002}. It's a pseudorandom function that gave the goal to break the monotony of a pattern and make it look more random.
Perlin Noise is the most known and used noise function. It was created by Ken Perlin, for the movie Tron 1982 with the aiming to generate natural looking textures.

The psedorandom property is important and a true random function like \emph{white noise} would not do the job. If we generate a texture based on white noise the pattern would change every time it's generated and we would like that the it stays the same, frame after frame. This is achieved with the use of inputs for this functions that with the same input returns always the same output sequence. 

The properties of an ideal \emph{noise} functions are as follows \cite{Ebert2002}:
\begin{itemize}
	\item \emph{noise} is a repeatable pseudorandom function of its inputs
	\item \emph{noise} has a known range, namely, from -1 to 1.
	\item \emph{noise} is band-limited, with a maximum frequency of about 1.
	\item \emph{noise} doesn't exhibit obvious periodicities or regular patterns. Such pseudorandom functions are always periodic, but the period can be made very long and therefore the periodicity is not conspicuous.
	\item \emph{noise} is \emph{stationary} - that is, its statistical character should be translationally invariant
	\item \emph{noise} is \emph{isotropic} - that is, its statistical character should be rotationally invariant
\end{itemize}

With this noise function, it's generated a sequence of values that are interpolated to generate a coherent noise. With the application of \emph{turbulence} that is composition of several layers of this noise with different frequencies and amplitudes forming a coherent noise. This layers are called \emph{Octaves} and the ratio between amplitude and frequency of the layers can be expressed as a constant known as \emph{persistence} \cite{Kelly2008}. With the result we can create a texture that looks natural and with fractal like structure.

\begin{figure}[htbp]
	\centering
	\includegraphics[width=0.55\textwidth]{img/Theory/Perlin_Noise/Merge.png}
	\caption{Different noise functions}
	\label{fig:merge}
\end{figure}

For instance, the Figure~\ref{fig:merge} shows the result of the interpolation over six noise functions with different frequencies and different amplitudes. And the sum of all this functions is the exibithed in the Figure~\ref{fig:noise} \cite{NoisesELIAS}.

\begin{figure}[htbp]
	\centering
	\includegraphics[width=0.55\textwidth]{img/Theory/Perlin_Noise/perlin1.png}
	\caption{``You may even imagine that it looks a little like a mountain range."}
	\label{fig:noise}
\end{figure}

Noise can also be used to generate planes. The method used is the same as the 1D problem but we have to generate a lot more data points that after are interpolated as a plane. This results in noisy images that are often used to model clouds, smoke and other textures with similar visual properties, Figure~\ref{fig:NTextures}. Another application for this technique is the generation of height maps like the one in figure .

\begin{figure}
        \centering
        \begin{subfigure}[b]{0.3\textwidth}
                \includegraphics[width=\textwidth]{img/Theory/Perlin_Noise/gradient_discrete.png}
                \label{fig:Fleaf}
        \end{subfigure}%
        ~ %add desired spacing between images, e. g. ~, \quad, \qquad, \hfill etc.
          %(or a blank line to force the subfigure onto a new line)
        \begin{subfigure}[b]{0.3\textwidth}
                \includegraphics[width=\textwidth]{img/Theory/Perlin_Noise/gradient_grey.png}
                \label{fig:Fbrocoli}
        \end{subfigure}
        ~ %add desired spacing between images, e. g. ~, \quad, \qquad, \hfill etc.
          %(or a blank line to force the subfigure onto a new line)
        \begin{subfigure}[b]{0.3\textwidth}
                \includegraphics[width=\textwidth]{img/Theory/Perlin_Noise/gradient_fire.png}
                \label{fig:Fbrocoli}
        \end{subfigure}
        \caption{Gradient mapped textures from \cite{NoisesGAMES}}
        \label{fig:NTextures}
\end{figure}

Another application for Noise planes are with \emph{object placement} on a grid. By creating a noise plane with the same size of the grid, with each cell of the grid corresponding to one pixel of noise, the object placement is done by choosing each object for each cell according to the noise value. Figure~\ref{fig:MyNCity}.
Figure~\ref{fig:NCity} shows a city which the buildings where placed with the use of a noise plane. In this case the noise domain was splitted in three intervals, each one corresponds to one type of building (commercial, industrial or residential). After setting the type for one block, the system randomly chooses one from a set of buildings of that type.

\begin{figure}[htbp]
    \centering
    \includegraphics[width=0.55\textwidth]{img/Theory/Perlin_Noise/AppletNoName201501191604.png}
    \caption{Objects Placed following a noise function}
    \label{fig:MyNCity}
\end{figure}

\begin{figure}[htbp]
	\centering
	\includegraphics[width=0.55\textwidth]{img/Theory/Perlin_Noise/NoisyCity.jpg}
	\caption{Objects Placed with a noise plane from \cite{NoisesGAMES}}
	\label{fig:NCity}
\end{figure}

% subsubsection noise (end)

%%!TEX root = ../../report.tex

 \subsubsection{Voxels [NOT \dots]} % (fold)
 \label{ssub:voxels}

``Voxel is a combination of ``volume" and ``pixel" ". i. e. the equivalent to pixels in 3D. In other words, if a pixel is a tiny square that represents a part of an model in a plane (picture), a voxel is a cube that represents a part of a model in a space.
Because we usually what we want to model  have three dimensions, it's easier to visualize this models with voxels.
We can see the world as a gigantic 3D matrix, and the voxels represent what is inside each point in the matrix.

Given that this voxels are just a way to store information, we still need a way to visualize this information. And this is the \trikyd part with this approach. How to visualize surfaces out of this 3D boxes. Since most of the hardware has evolved and beed adapted to work with textured triangles, and also a big volume of the research is arround textured triangles problems.
(\dots?)
% subsubsection voxels (end) 

%%!TEX root = ../../report.tex


\subsection{Architectural Styles} % (fold)
\label{sub:architectural_styles}

Most buildings or other man made structures are classified by it's \emph{Architectural Style}. An architectural style is characterized by the features that make a building notable and historically identifiable. This style features are related to form, building materials, construction methods, etc. and change trough different places and over the years. 

Even with this constant change in the styles there are some rules that we have to followed. The most obvious one is the chronological time.
If we are modelling a city from the 16th century it's wrong if we add buildings with styles form the 19th century. And we have to think also in a simmilar way with the places, if it's an asian city we do not add buildings with a known european layout.

And this correctness in relation to the architectural styles is also an important concern to this work.

% subsection architectural_styles (end)

%%!TEX root = ../../report.tex


\subsection{Urban Planning [TODO]} % (fold)
\label{sub:urban_planning}

Urban planning is defined by the Encyclopædia Britannica as ``design and regulation of the uses of space that focus on the physical form, economic functions, and social impacts of the urban environment and on the location of different activities within it.''

(\dots)

% subsection urban_planinng (end)

% subsection procedural_modelling_techniques (end)

%!TEX root = ../../report.tex


\subsection{Other Techniques} % (fold)
\label{sub:other_techniques}

\subsubsection{Level of Detail} % (fold)
\label{ssub:level_of_detail}

Level of Detail (LOD) is a technique that is used to improve the performance of the graphic pipeline. This is done by managing the complexity of the objects representation relative to some indicator.
Within this indicators, the most common one is the distance of each object to the viewer. If an object is far from the viewer a decrease on the detail will not be noticed and will save computation. Other indicators can be the importance that is assigned for each object, relative speed or partial occlusion.

This concept is easy to understand and implement if we look at the example in the Figure~\ref{fig:LOD2}. In this figure there are five cylinders that have different detail according to the distance to the camera. In this case only the number of sides of the cylinder changes.

\begin{figure}[htbp]
	\centering
	\includegraphics[width=0.95\textwidth]{img/OpenGL/LOD2.png}
	\caption{LOD example}
	\label{fig:LOD2}
\end{figure}
% subsubsection level_of_detail (end)

\subsubsection{Occlusion Culling} % (fold)
\label{ssub:occlusion_culling}

Occlusion culling (OC) is another technique that is used to improve performance. It involves determining the faces that are not visible at each point to be removed from the pipeline.

This procedure is usually done automatically by the GPU and applied to occluded faces behind other objects or out of the viewing frustum.

If this concept is applied before the generation of the objects, and prevent the inclusion of large amounts of geometry through the pipeline, we can make a large improvement on performance. The Figure~\ref{fig:viewingRange} is a good example. Here only the buildings that are visible from the current point are generated.

% subsubsection occlusion_culling (end)


% subsection other_techniques (end)



% subsection graphic_tools (end)


%%!TEX root = ../../report.tex

\subsection{DirectX} % (fold)
\label{sub:directx}
Graphics API made by Microsoft to their platforms, both Windows OS and their XBox Gaming platforms. This API is only used with this platforms therefore is not compatible with the core concerns of Rosetta that is portability.

% subsection directx (end)
%%!TEX root = ../../report.tex

\subsection{Pict3D} % (fold)
\label{sub:pict3d}

Pict3D is a tool written in Typed Racket that provides a functional interface to rendering hardware. This interface provides a set of geometric primitives that allows the user to produce a large set of very different models.

This tool provides also visualization mechanisms. It takes advantage of the Racket environment, DrRacket and can show the models in it's REPL along with the usual window visualization.

The big issue with this tool is performance. DrRacket have known performance issues, because it instruments the code for debug by default and also the garbage collection system is sometimes slow. This tool also have the issues inherent to interprocess communication.


% subsection pict3d (end)
%%!TEX root = ../../report.tex

\subsection{Computer Aided Design} % (fold)
\label{sub:computer_aided_design}
Computer Aided Design (CAD) is the use of computers during the creation, modification and analysis of design work. 


% subsection computer_aided_design (end)
%%!TEX root = ../../report.tex


\subsection{CEPL} % (fold)
\label{ssec:cepl}

CEPL is made in LISP and it that aims to provide building materials for realtime graphics demos and games. Although Racket is a dialog of LISP this tool `` is in pre-alpha. Everything is subject to change and many bugs are present in the code.''\cite{CEPL_GIT} So this tool would not be a sure bet to use in this work.







% subsection cepl (end)





% subsection conclusion (end)

